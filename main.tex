% !TeX TS-program = xelatex

\documentclass{resume}
\ResumeName{徐柯淞}
%\usepackage{geometry,fancyhdr,enumitem,footmisc,hyperref,xeCJKfntef,xcolor,ctex}
\begin{document}

\ResumeContacts{
  (+31) 6-5870-5772,%
  \ResumeUrl{mailto:isksxu@gmail.com}{isksxu@gmail.com},%
  \ResumeUrl{https://www.linkedin.com/in/kesongxu/}{linkedin.com/kesongxu},% \footnote{下划线内容包含超链接。}
  微信号: isksxu%
}

\ResumeTitle

\section{教育经历}
\ResumeItem
[代尔夫特理工大学|硕士研究生]
{代尔夫特理工大学}
[\textnormal{飞行性能专业,航空航天学院|} 理学硕士]
[2020.09—2023.07(预计)]

\textbf{2023年应届生},主修课程包括飞行器空气动力学、计算流体力学、多学科优化设计、知识工程、气动弹性力学、航空发动机技术、风洞实验与分析等。毕业论文研究一种机尾边界层吸入式飞机的气动声学。

\ResumeItem
[复旦大学|本科生]
{复旦大学}
[\textnormal{飞行器设计与工程,航空航天系|} 工学学士]
[2016.09—2020.06]

\textbf{GPA: 3.39/4.0(专业前 15\%)},每年均获优秀学生奖学金,“挑战杯”大学生课外学术科技作品竞赛复旦大学一等奖。

毕业设计:基于人工神经网络方法的非常规构型商用客机设计

\section{工作经历}

\ResumeItem{中仿智能科技(上海)股份有限公司}
[飞行仿真软件实习生]
[2021.06—2021.10] 

\begin{itemize}
  \item 负责一款中空中航时无人机的气动建模和初步测试,通过AVL等气动计算软件实现初步气动设计,使用C++配合飞行摇杆进行飞行性能测试,熟悉掌握了Git工作流进行团队协作。
  \item 使用MATLAB及Simulink参与A320飞行模拟器的气动调参及自动驾驶项目,使其飞行性能符合QTG要求。
  \item 使用MATLAB及C++将加速度传感器测量数据转换为飞行模拟器摇杆输入值。
\end{itemize}

\section{项目经历}

\ResumeItem{机尾边界层吸入式飞机的气动声学研究}
[研究生毕业论文]
[2022.03 至今]
\begin{itemize}
  \item 使用MATLAB通过调用PowerFLOW软件API实现了模型批量化前处理和后处理。
  \item 在学院 hpc 上对机尾边界层吸入飞机进行数值模拟,将结果与实验数据对比,对该构型进行气动噪音研究。
\end{itemize}

\ResumeItem{直升机性能、稳定性和控制}
[研究生课程项目]
[2022.02 - 2022.05]

\begin{itemize}
  \item 对一款直升机进行主要参数估计,使用3自由度模型对其进行机动模拟并使用PID调参对其进行优化。
\end{itemize}

\ResumeItem{实验与模拟}
[研究生课程项目]
[2021.12 - 2022.04] 

\begin{itemize}
  \item 小组合作通过无量纲化等实验设计准则制订实验及后处理计划,利用学院低速风洞模拟一款螺旋桨飞机在不同攻角、侧滑角、发动机转速下的飞行性能,综合研究该飞机方向舵在单发失效等不同条件下的舵效变化。
  \end{itemize}

\ResumeItem{计算流体力学:离散化技术}
[研究生课程项目]
[2020.12 - 2021.05] 

\begin{itemize}
  \item 合作并分别使用 Python/MATLAB 编写二维不可压缩 Navier-Stokes 方程求解器,利用该模型模拟顶盖方腔流体至稳态,将结果与相关文献进行对比,初步了解计算流体力学中的离散化概念。
\end{itemize}

\ResumeItem{航空航天工程中的计算流体力学}
[研究生课程项目]
[2020.09 - 2020.11]
\begin{itemize}
  \item 使用 ANSYS ICEM CFD + CFX 对 Aerospatiale A 翼型进行数值模拟,研究了不同网格密度、湍流模型和求解精度对结果影响,了解了 CFD 基础理论和常用计算方法。
\end{itemize}

\ResumeItem{基于人工神经网络方法的非常规构型商用客机设计}
[本科毕业设计]
[2019.09 - 2020.04]

\begin{itemize}
  \item 使用 Creo, HyperMesh 和 Fluent 建立并批处理一种非常规构型飞机的参数化模型,进行数值模拟,并使用人工神经网络的方法对该气动模型进行了优化。
  \item 评分:A (4.0 / 4.0).
\end{itemize}

\ResumeItem{基于血流储备分数的颈动脉狭窄评估}
[复旦大学曦源项目]
[2018.06-2019.06]
\begin{itemize}
  \item 使用 HyperMesh 和 ANSYS CFD 基于 CT 影像和临床数据建立颈动脉的流固耦合模型,进行数值模拟,建立一种具有较高准确度的颈动脉狭窄 FFR\textsubscript{CT} 评估模型。
  \item 成果以共同第一作者发表中文核心期刊一篇, SCI论文两篇,代表成果:\textbf{Kesong Xu}, Long Yu, Jun Wan, Shengzhang Wang, Haiyan Lu, “\ResumeUrl{https://doi.org/10.1080/10255842.2019.1710741}{The Influence of the Elastic Modulus of the Plaque in Carotid Artery on the Computed Results of FFR\textsubscript{CT}}”, Computer Methods in Biomechanics and Biomedical Engineering, Volume 23, Issue 5, 2020.
  \item 项目成果获“挑战杯”全国大学生课外学术科技作品竞赛上海市一等奖。
\end{itemize}

\section[其他]{技能/证书及其他}
\begin{itemize}
  \item \textbf{技能}: MATLAB, Creo, HyperMesh, ANSYS CFD (ICEM, CFX, Fluent), PowerFLOW, MS Office, LaTex, C++
  \item \textbf{证书}: 托福99/英语六级(549),AutoCAD(初级),普通话(二级甲等)。
  \item \textbf{活动}: 复旦大学彩云支南协会太阳花志愿者 
\end{itemize}

\end{document}
